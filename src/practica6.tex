
Supongo que los algoritmos ordenan de menor a mayor. 

\begin{itemize}
  \item \textsc{Selection sort}: El algoritmo selecciona en cada paso el mínimo de la lista y lo posiciona 
  en la primer ubicación no ordenada. Es independiente de la entrada por lo cual será $O(n^2)$.
  \item \textsc{Insertion sort}: El mejor caso del algoritmo es cuando el arreglo ya está ordenado. Si el 
  arreglo estaba ordenado de mayor a menor, en cada iteración el elemento en la posición i
  deberá ser swapeado i veces  ya que será el nuevo mínimo del sub-arreglo ya ordenado. Por ende la complejidad
  será $O(\sum\limits_{i=0}^{n}i)$ = $O(n^{2})$ donde n es el tamaño del arreglo.
  \item \textsc{Heap sort}: DESPUES COMPLETAR.
  \item \textsc{Quicksort}: El tiempo de ejecución del algoritmo depende de la elección del pivote en cada 
  iteración. Por ende la complejidad en el peor caso será cuando la elección del pivote en cada iteración sea 
  COMPLETAR LUEGO.
  \item \textsc{Mergesort}: La ecuación de recurrencia del algoritmo es $T(n) = T(n/2) + O(n)$ por lo cual 
  el algoritmo tendrá complejidad temporal $O(n log n)$ independientemente de la manera en la que esté el 
  arreglo.  
\end{itemize}

----------------------

Si Tam(s') >> Tam(s'') o Tam(s') ~ Tam(s'')
Utilizaría el Insertion sort ya que al estar ordenado los primeros Tam(s') elementos ordenados las primeras 
Tam(s') iteraciones costarán $O(1)$. La i-esima iteracion de un elemento en s'' costara Tam(s') + i
por ende $O(\sum\limits_{i=0}^{Tam(s'')}Tam(s')+ i)$ = $O(Tam(s'')*Tam(s'))$ lo cual conviene si Tam(s') >> Tam(s'')

Si Tam(s') << Tam(s'') usaría alguno de los algoritmos que tenga $O(n log n)$ complejidad temporal, donde 
n = Tam(s'').

-------------------------

\algoritmo{MásChicos}{in/ A : arreglo, in/ k : nat}{res: arreglo(nat)}{
  \State $\var{res} \larr CrearArreglo(k)$
  \complejidad{O($k$)}
  \State $\var{presentes} \larr CrearArreglo(tam(A))$
  \complejidad{O($n$)}
  \State $\var{max} \larr 1$
  \complejidad{O(1)}

  \For {$i = 2 to Tam(A)$}
  \complejidad{O($n$)}
    \If {$A[i] > A[max]$}
    \complejidad{O(1)}
      \State $max \larr i$
    \EndIf
  \EndFor

  \For {$i = 1 to Tam(A)$}
  \complejidad{O($n$)}
    \State $presentes[i]$
  \EndFor

  \For {$i = 1 to k$}
  \complejidad{O($n$)}
    \State $\Var{min} \larr max$
    \For {$j = 1 to Tam(A)$}
      \If {$presentes[j] \wedge A[j] < A[min]$}
      \complejidad{O(1)}
        \State $min \larr j$
        \complejidad{O(1)}
      \EndIf
    \EndFor
    \State $presentes[min] \larr false$
    \State $res[i] \larr A[min]$
  \EndFor
}{O($n+n+nk$)}

Por ende el algoritmo tiene complejidad O($nk$) con n = Tam(A). Como la cota inferior para ordenar un arreglo
es $\theta$(n log n), con lo cual el algoritmo tiene mejor complejidad cuando k es de orden menor que log n

Presentes indica si ya incluimos un elemento entre la lista de los k mas chicos.

--------------------------

\algoritmo{MuchosMerge}{in/ S : conj(arreglo(nat)), in/ n : nat}{res: arreglo(nat)}{
  \State $\var{res} \larr s_1$
  \complejidad{O($1$)}

  \For {$i = 2 to n$}
  \complejidad{O($n$)}
    \State $Merge(res, res, s_i)$
    \complejidad{O($max\{Long(s_i), res\}$)}
  \EndFor

}{$O(n*max_{i}(s_i))$}

Al finalizar $res$ contiene el resultado de hacer Merge entre todas las listas del conjunto. Como Merge devuelve un arreglo ordenado y contiene la unión de las 2 listas que recibe, res = $\cup\limits_{1=1}^{n}(s_i)$.

--------------------------

\algoritmo{PorFrecuencia}{in/out A : arreglo(nat)}{res: arreglo(nat)}{
  \State $\var{dicc_{AVL}} d \leftarrow Vacío$
  \State $\var{n} \leftarrow Tam(a)$
  \complejidad{O($1$)}

  \For {$i = 1 to n$}
  \complejidad{O($n*log(n)$)}
    \IF {$\neg$Definido?(d, A[i])}
      \STATE $Definir(d, A[i], 1)$
      \complejidad{O($log(n)$)}
    \ELSE
      \STATE $Definir(d, A[i], obtener(d, A[i])+1)$
      \complejidad{O($log(n)$)}
    \EndIF
  \EndFor

  \For {$i = 1 to n$}
  \complejidad{O($n*log(n)$)}
    \STATE $max \leftarrow DameMax(d)$
    \complejidad{O($d$)}
    \STATE $repeticiones \leftarrow obtener(d, max)$
    \For {$i = 1 to repeticiones$}
      \complejidad{O($repeteciones$)}
      \STATE $A[i] = max$
    \EndFor
    \STATE $Borrar(d, max)$
    \complejidad{O($n$)}
  \EndFor


}{$O(n*log(n))$}

La idea del algoritmo es tener un diccionario AVL que determina la relación de órden a partir de su significado. La clave del diccionario Avl son los elementos del arreglo $A$ y el significado la cantidad 
de repeticiones. Luego se tiene que ir sacando el máximo del Avl (el elemento con mayor cantidad de repeticiones del arreglo) y posicionarlo esa cantidad de veces en el arreglo original.

-------------------

\algoritmo{AvlSort}{in/out A : arreglo(nat)}{res: arreglo(nat)}{
  \State $\var{dicc_{AVL}} d \leftarrow Vacío$
  \State $\var{n} \leftarrow Tam(a)$
  \complejidad{O($1$)}

  \For {$i = 1 to n$}
  \complejidad{O($n*log(n)$)}
    \IF {$\neg$Definido?(d, A[i])}
      \STATE $Definir(d, A[i], 1)$
      \complejidad{O($log(n)$)}
    \ELSE
      \STATE $Definir(d, A[i], obtener(d, A[i])+1)$
      \complejidad{O($log(n)$)}
    \EndIF
  \EndFor

  \For {$i = 1 to n$}
  \complejidad{O($n*log(n)$)}
    \STATE $min \leftarrow DameMin(d)$
    \complejidad{O($d$)}
    \STATE $repeticiones \leftarrow obtener(d, min)$
    \complejidad{O($d$)}
    \For {$i = 1 to repeticiones$}
      \complejidad{O($repeteciones$)}
      \STATE $A[i] = max$
    \EndFor
    \STATE $Borrar(d, max)$
    \complejidad{O($d$)}
  \EndFor


}{$O(n*log(n))$}

La idea del algoritmo es tener un diccionario AVL que determina la relación de órden a partir de las claves. La clave del diccionario Avl son los elementos del arreglo $A$ y el significado la cantidad 
de repeticiones. Luego se tiene que ir sacando el mínimo del Avl (el elemento más chico del arreglo) y posicionarlo esa cantidad de veces en el arreglo original.

-------------------------

\algoritmo{OrdenarPlanilla}{in/out A : arreglo(alumno)}{res: arreglo(alumno)}{
  \State $\var{lista(alumno)} varones \leftarrow Vacía$
  \complejidad{O($1$)}
  \State $\var{lista(alumno)} mujeres \leftarrow Vacía$
  \complejidad{O($1$)}
  \State $\var{n} \leftarrow Tam(a)$
  \complejidad{O($1$)}

  \For {$i = 1 to n$}
  \complejidad{O($n$)}
    \IF {A[i].sexo = masc}
      \STATE $AgregarAtras(varones, A[i])$
      \complejidad{O(1)}
    \ELSE
      \STATE $AgregarAtras(mujeres, A[i])$
      \complejidad{O(1)}
    \EndIF
  \EndFor

  \STATE $BucketSort(varones)$
  \complejidad{O($n$)}
  \STATE $BucketSort(mujeres)$
  \complejidad{O($n$)}

  \STATE $Concatenar(mujeres, varones)$

}{$O(n)$}

2) En un primer paso se divide la lista inicial en dos, donde cada una contiene un único género. Esto mantiene el orden relativo.
Luego, cuando se realiza el BucketSort, al ser un ordenamiento estable, se puede garantizar que el orden relativo entre dos elementos iguales (por su relación de orden) se mantiene.

3) No, es todo una mentira.

---------------------------

\algoritmo{MuchosCasiSort}{in/out A : arreglo(nat)}{res: arreglo(nat)}{
  \State $\var{nat} i \leftarrow 1$
  \complejidad{O($1$)}
  \State $\var{nat} i \leftarrow 0$
  \complejidad{O($1$)}
  \State $\var{n} \leftarrow Tam(a)$
  \complejidad{O($1$)}

  \While {$j < log_{2}(n)$}
  \complejidad{O($n*log(n)$)}
    \STATE $CasiSort(subArreglo(A, i, n))$
    \complejidad{O($n$)}
    \STATE $j \leftarrow j + 1$
    \complejidad{O(1)}
    \STATE $i \leftarrow i + (n/2^{j})$
    \complejidad{O(1)}
  \EndWhile

  \While {$j > 0$}
  \complejidad{O($n*log(n)$)}
    \STATE $i \leftarrow i - (n/2^{j})$
    \complejidad{O(1)}
    \STATE $Merge(subArreglo(A, i, n), subArreglo(A, i, (i+n)/2), subArreglo(A, (i+n)/2, n))$
    \complejidad{O($n$)}
  \EndWhile

}{$O(n)$}
